\documentclass{aastex62}

\usepackage{amsthm, amsmath, amssymb}
\usepackage{latexsym,graphicx,rotating,amsmath, epsfig, natbib, graphbox}
\usepackage{listings}

\newcommand{\sol}{\odot}
\newcommand{\del}{\nabla}
\newcommand{\cross}{\times}
\newcommand{\avg}{\bar}
\renewcommand{\vec}{\boldsymbol}
\newcommand{\pomega}{\varpi}
\newcommand{\conv}{\boldsymbol}

\newcommand{\scrD}{\mathcal{D}}
\newcommand{\scrH}{\mathcal{H}}
\newcommand{\scrR}{\mathcal{R}}
\newcommand{\scrL}{\mathcal{L}}
\newcommand{\scrS}{\mathcal{S}}

\newcommand{\Ra}{\mathrm{Ra}}
\newcommand{\Ek}{\mathrm{Ek}}
\renewcommand{\Pr}{\mathrm{Pr}}
\newcommand{\Pm}{\mathrm{Pm}}
\newcommand{\RoCsq}{\mathrm{Ro}_\mathrm{C}^2}
\newcommand{\RoC}{\mathrm{Ro}_\mathrm{C}}

\newcommand{\dedalus}{\href{http://dedalus-project.org/}{Dedalus}}

\definecolor{codegreen}{rgb}{0,0.6,0}
\definecolor{codegray}{rgb}{0.5,0.5,0.5}
\definecolor{codepurple}{rgb}{0.58,0,0.82}
\definecolor{backcolour}{rgb}{0.95,0.95,0.92}
\definecolor{codered}{rgb}{0.6,0,0}
\definecolor{codeblue}{rgb}{0,0,0.6}

% \watermark{text}
\begin{document}
\section{Basic equations}
From Geoff Vasil's "gauged" document, the fully compressible equations are:
\begin{equation}
  \partial_t \ln \rho + \scrD_1 \cdot \vec{u} = 0
\end{equation}
\begin{equation}
  \partial_t \vec{u} + \vec{u}\cdot \del \vec{u} + \del (h + \phi) = T\del s + \vec{\scrD}_1\cdot(\nu E)
\end{equation}
\begin{equation}
  \partial_t s + \vec{u}\cdot \del s = \frac{1}{T}\left[\vec{\scrD}_1 (\chi \del h) + \frac{\nu}{2}\mathrm{Tr}(E^2)\right]
\end{equation}
with
\begin{equation}
  T = \frac{h}{c_P}, \quad E = \del \vec{u} + (\del \vec{u})^\mathrm{T} - \frac{2}{3}(\del\cdot\vec{u})\mathrm{I},
\end{equation}
where
\begin{equation}
  \vec{\scrD}_g = \del + g \del \ln \rho,
\end{equation}
and linked by an ideal gas equation of state:
\begin{equation}
  \frac{\gamma}{\gamma-1} \frac{s}{c_P} - \frac{1}{\gamma - 1}\ln h + \ln \rho =0
\end{equation}

\subsection{Enthalpy all the time}
The first step is to rewrite the entropy equation to be entirely in terms of enthalphy; the diffusive term suggests a log enthalphy form:
\begin{equation}
  \frac{1}{h} \vec{\scrD}_1 \cdot (\chi \del h) =
  \del\cdot(\chi \ln h) + \chi (\del \ln h)^2 + \chi \ln \rho \cdot \ln h
  = \vec{\scrD}_1 \cdot (\chi \ln h) + \chi (\del \ln h)^2
\end{equation}
and
\begin{equation}
\frac{1}{c_P}\left(\partial_t s + \vec{u}\cdot \del s\right) = \vec{\scrD}_1 \cdot (\chi \ln h) + \chi (\del \ln h)^2 + \exp{(-\ln h)}\frac{\nu}{2}\mathrm{Tr}(E^2).
\end{equation}
We might be alarmed at trading a $1/T$ for a $\exp{(-\ln h)}$, but we shouldn't be: $1/T$ is already a band-unlimited nonlinear term, and $\mathrm{Tr}(E^2)$ is always going to be on the RHS.

The form above is best if $\chi$ is constant in time.  If we instead take $\kappa$ to be constant in space and time, then the diffusive term is:
\begin{align}
  \frac{1}{\rho T}\del\cdot\kappa \del T &= \frac{\kappa}{\rho}\frac{1}{T}\del\cdot\del T = \frac{\kappa}{\rho}\left[\nabla^2 \ln T + \left(\nabla \ln T\right)^2\right] \\
  &=
  \frac{\kappa}{\rho}\left[\nabla^2 \ln h + \left(\nabla \ln h\right)^2\right]
\end{align}
or, taking $\mu = \nu \rho$:
\begin{equation}
  \left(\partial_t \frac{s}{c_P} + \vec{u}\cdot \del \frac{s}{c_P}\right) = \frac{\kappa}{\rho}\left[\nabla^2 \ln h + \left(\nabla \ln h\right)^2\right] + \exp{(-\ln h)}\exp{(-\ln \rho)}\frac{\mu}{2}\mathrm{Tr}(E^2),
\end{equation}
which now has two exponentials in the viscous heating term, but again that term is always going to be RHS anyways.  We've also collected the s and $c_P$ terms together.

The momentum equation takes this form:
\begin{equation}
  \partial_t \vec{u} + \vec{u}\cdot \del\vec{u} + \del (\exp{(\ln h)} + \phi) = \exp{(\ln h)}\frac{\del s}{c_P} + \vec{\scrD}_1\cdot(\nu E),
\end{equation}
or
\begin{equation}
  \partial_t \vec{u} + \vec{u}\cdot \del\vec{u} + \del (\exp{(\ln h)} + \phi) = \exp{(\ln h)}\frac{\del s}{c_P} + \exp{-\ln\rho}\mu \del\cdot(E),
\end{equation}
for constant $\mu$.
Clearly we're going to have to be a bit careful with our now nonlinear pressure gradient term (buoyancy has also changed from a quadratic nonlinearity to a band-unlimited nonlinearity).  To make progress, let's consider equilibria.

\subsection{Hydrostatic equilibrium in a polytrope}
If we assert hydrostatic equilbrium, we have:
\begin{equation}
  \del (h + \phi) = h\frac{\del s}{c_P}
\end{equation}
or
\begin{equation}
  \del \ln h  - \frac{\del s}{c_P} = - \frac{\del \phi}{h}
\end{equation}
For an ideal gas we take
\begin{equation}
  \frac{\del s}{c_P} = \frac{1}{\gamma}\del \ln h - \frac{\gamma-1}{\gamma}\del \ln \rho
\end{equation}
and for a polytrope, we take
\begin{equation}
  \ln \rho = m \ln h
\end{equation}
in place of the thermal equation.

If we take $L = R T_c/\phi_c$, or the density scale height of the equivalent isothermal atmosphere at the characteristic level (say the bottom of the atmosphere), take constant gravity, and take $s_c = c_P$, then we get this non-dimensional form:
\begin{equation}
  \del \ln h  - \del s = - \frac{\gamma-1}{\gamma}\exp{(-\ln h)} \vec{\hat{z}}
\end{equation}
where the $c_P$ in the enthalpy has been factored to pull out the $R$ separately from the $(\gamma-1)/\gamma$

You know, I don't think there is actually an lengthscale information here.  What we have is $R T_c/\phi_c = 1$, because if L is the characteristic lengthscale for $\del \phi$, then the L's cancel out.  Does that somehow imply that $L=H_\rho$?  Alternatively, the correct scale is $\del \phi = \vec{g}$ and then we have a lengthscale $ L = R T_c/g_c = H_\rho$.  Very confusing.

I wonder also if there's anything in this ratio:
\begin{equation}
  \frac{\phi_c}{h_c}~\text{or}~L=\frac{g}{h_c}
\end{equation}
I guess that's an adiabatic enthalphy scale height?
(based on:)
\begin{equation}
  \del \ln h = - \frac{\del \phi}{h}, \quad \del s = 0.
\end{equation}
For now we stick to a density scaleheight system, just because.

Here's some code that solves this above problem:
\lstset{language=Python,
  backgroundcolor=\color{backcolour},
  commentstyle=\color{codered},
  keywordstyle=\color{codepurple},
  numberstyle=\tiny\color{codegray},
  stringstyle=\color{codepurple},
  frame=lines,
  breakatwhitespace=false,
  breaklines=true,
  captionpos=b,
  keepspaces=true,
  numbers=left,
  numbersep=5pt,
  showspaces=false,
  showstringspaces=false,
  showtabs=false,
  tabsize=2,
  inputencoding=utf8,
  extendedchars=true,
  literate={Υ}{{$\Upsilon$}}1 {θ}{{$\theta$}}1 {τ}{{$\tau$}}1 {φ}{{$\phi$}}1 {γ}{{$\gamma$}}1,
  }
\begin{lstlisting}
HS_problem = problems.NLBVP([Υ, θ, S, τθ])
HS_problem.add_equation((grad(θ) - grad(S) + τθ*P1 ,
                         -1*exp(-θ)*grad_φ*ez))
HS_problem.add_equation((Υ - m*θ, 0))
HS_problem.add_equation((S - 1/γ*θ+(γ-1)/γ*Υ, 0))
HS_problem.add_equation((θ(z=0),0))
\end{lstlisting}

\subsection{Exact solutions to polytropes}
The exact solution to a polytrope is:
\begin{equation}
  h(z) = h(z=0) + h_z z
\end{equation}
where
\begin{equation}
  h(z=0) =
  \begin{cases}
    1, \text{if}\ z_c=0\\
    1+L_z h_z, \text{if}\ z_c=L_z
  \end{cases}
\end{equation}
and if $L = (h/g)(z=z_c) = H_h(z=z_c)$:
\begin{equation}
\del_\phi = 1 \quad\text{and}\quad
h_z = -\frac{\gamma}{\gamma-1}\frac{1}{(m+1)}
\end{equation}
but if $L = (RT/g)(z=z_c) = H_\rho(z=z_c)$:
\begin{equation}
\del_\phi = \frac{(\gamma-1)}{\gamma} \quad\text{and}\quad
h_z = -\frac{1}{(m+1)}
\end{equation}


\section{A thermal equilibrum state}
Let's now assert a base state, assuming hydrostatic equilibrium and a thermal equilibrium.

Recognizing that:
\begin{equation}
  \exp{(\ln h)} = \exp{(\ln h_0)}\exp{(\ln h_1)} = h_0 \exp{(\ln h_1)}
\end{equation}
and that:
\begin{equation}
  \del \left[h_0 \exp{(\ln_1)} + \phi\right] = \del \left[h_0 \exp{(\ln_1)-1} +h_0 + \phi\right]=  \del\left[h_0\left(\exp{(\ln h_1)}-1\right)\right]
\end{equation}
we have:
\begin{equation}
  \partial_t \vec{u} + \vec{u}\cdot \del\vec{u} + \del\left[h_0\left(\exp{(\ln h_1)}-1\right)\right] = \frac{h_0}{c_P}\exp{(\ln h_1)}\del s + \frac{h_0}{c_P}\left(\exp{(\ln h_1)}-1\right)\del s_0 + \frac{\mu}{\rho}\vec{\scrD}_1\cdot(E),
\end{equation}
or
\begin{equation}
  \partial_t \vec{u} + \vec{u}\cdot \del\vec{u} + \del\left[h_0\left(\exp{(\ln h_1)}-1\right)\right] -h_0 \ln h_1 \frac{\del s_0}{c_P} = h_0\exp{(\ln h_1)}\frac{\del s}{c_P} + h_0\left(\exp{(\ln h_1)}-1-\ln h_1\right)\frac{\del s_0}{c_P} + \frac{\mu}{\rho}\vec{\scrD}_1\cdot(E),
\end{equation}
or, going to $\theta = \ln h_1$ and absorbing $c_P$ into $s$:
\begin{equation}
  \partial_t \vec{u} + \vec{u}\cdot \del\vec{u} + \del\left[h_0\theta\right] -h_0 \theta \del s_0 = - \del\left[h_0\left(\exp{(\theta)}-1-\theta\right)\right] + h_0\exp{(\theta)}\del s + h_0\left(\exp{(\theta)}-1-\theta\right)\del s_0 + \frac{\mu}{\rho}\vec{\scrD}_1\cdot(E),
\end{equation}

If we non-dimensionalize on a characteristic velocity $u_c$, then we're seeing the following non-dimensional parameter:
\begin{equation}
  \frac{h_0}{u_c^2} = \left(\frac{\gamma}{\gamma-1}\right)\left(\frac{R T}{u_c^2}\right) = \left(\frac{\gamma}{\gamma-1}\right) \mathrm{Ma}^2
\end{equation}
for an isothermal Mach number $\mathrm{Ma}$.  This suggests:
\begin{equation}
  \partial_t \vec{u} + \vec{u}\cdot \del\vec{u} + \left(\frac{\gamma}{\gamma-1}\right) \mathrm{Ma}^2 \del\left[h_0\left(\exp{(\ln h_1)}-1\right)\right] = \mathrm{Ma}^2 h_0 \exp{(\ln h_1)}\del s + \vec{\scrD}_1\cdot(\nu E),
\end{equation}
with $h_0$ now scaled to some reference value.

\emph{A note on computation:} the $\left(\exp{(\ln h_1)}-1\right)$ term should be computed using \verb+numpy.expm1(x)+ rather than computing \verb+numpy.exp(x)+ and then subtracting 1, especially when $x \sim \mathrm{Ma}^2 \ll 1$ (numerical convergence issues).

\textbf{To do: add np.expm1 to the UnaryGridFunction list.}

There's one more trick that's likely useful for low-Mach settings:
\begin{equation}
  \partial_t \vec{u} + \vec{u}\cdot \del\vec{u} + \left(\frac{\gamma}{\gamma-1}\right) \mathrm{Ma}^2 \del\left[h_0 \ln h_1 \right] = \mathrm{Ma}^2 h_0 \exp{(\ln h_1)}\del s -
  \left(\frac{\gamma}{\gamma-1}\right) \mathrm{Ma}^2 \del\left[h_0\left(\exp{(\ln h_1)}-1-\ln h_1\right)\right] + \vec{\scrD}_1\cdot(\nu E),
\end{equation}
or
\begin{equation}
  \partial_t \vec{u} + \vec{u}\cdot \del\vec{u} + \left(\frac{\gamma}{\gamma-1}\right) \mathrm{Ma}^2 \del\left[h_0 \ln h_1 \right] = \mathrm{Ma}^2 h_0 \exp{(\ln h_1)}\del s -
  \left(\frac{\gamma}{\gamma-1}\right) \mathrm{Ma}^2 \del\left[h_0\left(\mathrm{expm1}{(\ln h_1)}-\ln h_1\right)\right] + \vec{\scrD}_1\cdot(\nu E),
\end{equation}
where we've just shifted a likely wave-like linear term to the LHS.  Not clear there is a similar one in the quadratic thermal nonlinearity, but that's also because we assumed $\del s_0 = 0$, and there are no gravity waves in that basic state (just acoustic).  That's actually very interesting.


\section{Here we try an adiabatic state}

\subsection{Some kind of equilibrium}
Let's now assert a base state, assuming hydrostatic equilibrium and an adiabatic profile.  This differs from assuming thermal equilibrium.
Let:
\begin{equation}
  \del s_0 = 0
\end{equation}
and
\begin{equation}
  \del(h_{0,0}\exp{\ln h_0} + \phi) = 0
\end{equation}
this means:
\begin{equation}
  h_{0,0}\exp{\ln h_0} + \phi = h_0 + \phi = \scrH
\end{equation}
for some gauge constant $\scrH$.

\subsection{Momentum equation about HS eq}

What's the momentum equation look like for fluctuations about this hydrostatic, adiabatic equilibrium?

Recognizing that:
\begin{equation}
  \exp{(\ln h)} = \exp{(\ln h_0)}\exp{(\ln h_1)} = h_0 \exp{(\ln h_1)}
\end{equation}
and that:
\begin{equation}
  \del \left[h_0 \exp{(\ln_1)} + \phi\right] = \del\left[h_0\left(\exp{(\ln h_1)}-1\right)\right]
\end{equation}
we have:
\begin{equation}
  \partial_t \vec{u} + \vec{u}\cdot \del\vec{u} + \del\left[h_0\left(\exp{(\ln h_1)}-1\right)\right] = \frac{h_0}{c_P}\exp{(\ln h_1)}\del s + \vec{\scrD}_1\cdot(\nu E),
\end{equation}
If we non-dimensionalize on a characteristic velocity $u_c$, then we're seeing the following non-dimensional parameter:
\begin{equation}
  \frac{h_0}{u_c^2} = \left(\frac{\gamma}{\gamma-1}\right)\left(\frac{R T}{u_c^2}\right) = \left(\frac{\gamma}{\gamma-1}\right) \mathrm{Ma}^2
\end{equation}
for an isothermal Mach number $\mathrm{Ma}$.  This suggests:
\begin{equation}
  \partial_t \vec{u} + \vec{u}\cdot \del\vec{u} + \left(\frac{\gamma}{\gamma-1}\right) \mathrm{Ma}^2 \del\left[h_0\left(\exp{(\ln h_1)}-1\right)\right] = \mathrm{Ma}^2 h_0 \exp{(\ln h_1)}\del s + \vec{\scrD}_1\cdot(\nu E),
\end{equation}
with $h_0$ now scaled to some reference value.

\emph{A note on computation:} the $\left(\exp{(\ln h_1)}-1\right)$ term should be computed using \verb+numpy.expm1(x)+ rather than computing \verb+numpy.exp(x)+ and then subtracting 1, especially when $x \sim \mathrm{Ma}^2 \ll 1$ (numerical convergence issues).

\textbf{To do: add np.expm1 to the UnaryGridFunction list.}

There's one more trick that's likely useful for low-Mach settings:
\begin{equation}
  \partial_t \vec{u} + \vec{u}\cdot \del\vec{u} + \left(\frac{\gamma}{\gamma-1}\right) \mathrm{Ma}^2 \del\left[h_0 \ln h_1 \right] = \mathrm{Ma}^2 h_0 \exp{(\ln h_1)}\del s -
  \left(\frac{\gamma}{\gamma-1}\right) \mathrm{Ma}^2 \del\left[h_0\left(\exp{(\ln h_1)}-1-\ln h_1\right)\right] + \vec{\scrD}_1\cdot(\nu E),
\end{equation}
or
\begin{equation}
  \partial_t \vec{u} + \vec{u}\cdot \del\vec{u} + \left(\frac{\gamma}{\gamma-1}\right) \mathrm{Ma}^2 \del\left[h_0 \ln h_1 \right] = \mathrm{Ma}^2 h_0 \exp{(\ln h_1)}\del s -
  \left(\frac{\gamma}{\gamma-1}\right) \mathrm{Ma}^2 \del\left[h_0\left(\mathrm{expm1}{(\ln h_1)}-\ln h_1\right)\right] + \vec{\scrD}_1\cdot(\nu E),
\end{equation}
where we've just shifted a likely wave-like linear term to the LHS.  Not clear there is a similar one in the quadratic thermal nonlinearity, but that's also because we assumed $\del s_0 = 0$, and there are no gravity waves in that basic state (just acoustic).  That's actually very interesting.

\subsection{Making a super-adiabatic atmosphere}
If the zero state has $\del s_0 = 0$, then the superadiabaticity must be in the initial conditions.  Here's what that looks like.

The hydrostatically balanced temperature profile is:
\begin{equation}
  T = \frac{g}{c_P} \ldots
\end{equation}

\subsection{Thermal considerations}



The lack of thermal equilibrium implies a constant heating source Q:
\begin{equation}
  Q = \vec{\scrD}_1 \cdot (\chi \ln h_0) + \chi (\del \ln h_0)^2
\end{equation}

The equation set becomes:
\begin{equation}
  \partial_t \vec{u} + \vec{u}\cdot \vec{u} + \del (h_0\left[\exp{(\ln h_1)}-1\right]) = \frac{1}{c_P}h_0(\exp{(\ln h_1)})\del s_1 + \vec{\scrD}_1\cdot(\nu E),
\end{equation}
or
\begin{equation}
  \partial_t \vec{u} + \vec{u}\cdot \vec{u} + \del (h_0 \ln h_1) - \frac{1}{c_P}h_0 \del s_1 = -\del (h_0\left[\exp{(\ln h_1)}-1-\ln h_1 \right]) + \frac{1}{c_P}h_0(\exp{(\ln h_1)} - 1)\del s_1 + \vec{\scrD}_1\cdot(\nu E),
\end{equation}
where we have de-stiffened both the nonlinear pressure gradient term and the nonlinear buoyancy term.
The viscous term is:
\begin{equation}
  \vec{\scrD}_1\cdot(\nu E) = (\del + \del \ln \rho_0 + \del \ln \rho_1) \cdot(\nu E)
\end{equation}
so:
\begin{equation}
  \partial_t \vec{u} + \vec{u}\cdot \vec{u} + \del (h_0 \ln h_1) - \frac{1}{c_P}h_0 \del s_1 - \vec{\scrD}_{1,0} \cdot(\nu E) = -\del (h_0\left[\exp{(\ln h_1)}-1-\ln h_1 \right]) + \frac{1}{c_P}h_0(\exp{(\ln h_1)} - 1)\del s_1 + \del \ln \rho_1\cdot(\nu E),
\end{equation}
We need some better notation.  Let $\Theta = \ln h$ and $\Upsilon = \ln \rho$.
\begin{equation}
  \partial_t \vec{u} + \vec{u}\cdot \vec{u} + \del (h_0 \Theta_1) - \frac{1}{c_P}h_0 \del s_1 - \vec{\scrD}_{1,0} \cdot(\nu E) = -\del (h_0\left[\exp{\Theta_1}-1-\Theta_1 \right]) + \frac{1}{c_P}h_0(\exp{\Theta_1} - 1)\del s_1 + \del \Upsilon_1\cdot(\nu E),
\end{equation}

For the entropy equation, we need to decompose the RHS:
\begin{align}
\vec{\scrD}_1 \cdot (\chi \Theta) + \chi (\del \Theta)^2 &=
\vec{\scrD}_1 \cdot (\chi \Theta_0) + \chi (\del \Theta_0)^2
+ \vec{\scrD}_1 \cdot (\chi \Theta_1) + \chi (\del \Theta_1)^2
+ 2 \chi (\del \Theta_0\cdot \del \Theta_1) \\
& = \vec{\scrD}_{1,0} \cdot (\chi \Theta_0) + \chi (\del \Theta_0)^2 \\
& \phantom{=} + \del \Upsilon_1 \cdot(\chi \Theta_0)
+ \vec{\scrD}_{1,0} \cdot (\chi \Theta_1) \\
& \phantom{=} + \del \Upsilon_1 \cdot(\chi \Theta_1)+ \chi (\del \Theta_1)^2
 + 2 \chi (\del \Theta_0\cdot \del \Theta_1) \\
& = Q + \del \Upsilon_1 \cdot(\chi \Theta_1)+ \chi (\del \Theta_1)^2
+ 2 \chi (\del \Theta_0\cdot \del \Theta_1) \\
& \phantom{=} + \del \Upsilon_1 \cdot(\chi \Theta_0)
+ \vec{\scrD}_{1,0} \cdot (\chi \Theta_1)
\end{align}
the entropy equation is:
\begin{equation}
\frac{1}{c_P}\left(\partial_t s_1 + \vec{u}\cdot \del s_1\right) =
Q +
 \vec{\scrD}_1 \cdot (\chi \ln h) + \chi (\del \ln h)^2 + \exp{(-\ln h)}\frac{\nu}{2}\mathrm{Tr}(E^2).
\end{equation}

Our variables are now:
\begin{equation}
\ln h = \ln h_0 + \ln h_1, \quad ln \rho = \ln \rho_0 + \ln \rho_1, \quad s = s_0 + s_1.
\end{equation}

Thermal equilibrium means:




\end{document}
